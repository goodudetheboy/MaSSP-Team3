\documentclass{article}
\usepackage{tabto}
\usepackage{graphicx}

\title{\textbf{MaSSP AI PROJECT DAILY REPORT NO. 3}}
\author{\textbf{Team 3}\\
Ho Chi Vuong\\
Nim Tri Nghia\\
Nguyen Khac Minh}
\date{Wednesday, July 10th, 2019}


\newcommand{\code}[1]{\texttt{#1}}

\begin{document}
\maketitle
\textit{Project theme: OBJECT DETECTION – Finding certain objects from input images or videos} 
\graphicspath{ {./images/} }

\section{General Progress }
\begin{itemize}
	\item Decided to build both a Website and a Desktop application for our AI model.
	\item Successfully implemented YOLO model on realtime object detection through camera using OpenCV library
	\item Trying to implemented the GUI for the app
	\item Decided \textit{not} to retrain the YOLO model as the pre-trained one already covers most of what we have in mind
\end{itemize}
	


\section{Future Plan}

\begin{itemize}
	\item Finish making a server using Minh's computer as the host to run the code on, 
	\item Wrap up the desktop application
	\item Finish the technical report and presentation slides for the mock presentation day
\end{itemize}

\section{Obstacles}
\begin{itemize}
	\item Minor problem while developing the website like security error and such
	\item \code{UnknownError} arise during the re-tweaking of our code (e.g. System exit 2, \code{AttributeError})
	\begin{itemize}
		\item Sometimes, when we try to run two .ipynb file at the same time, both do not run and exited with \code{AttributeError}
	\end{itemize}
	\item We are unable to run the desktop application package that mentors send us (details in section 4.1)
\end{itemize}

\section{Demo}

The Colab link containing the progress so far: \emph{\underline{https://drive.google.com/open?id=}}\\
\emph{\underline{186OWo8ZU0dRYfMjsqumBpMVNRWlR2kwh}} as well as our team's repository on GitHub: \emph{\underline{https://github.com/goodudetheboy/MaSSP-Team3}}.

\subsection{AI Core Analysis}
	\begin{itemize}
		\item Using the \code{detect\_video} function extracted from \code{yolo.py} file from the \code{keras-yolo-master}, we tweaked it to fit into our code and made it into \code{detect\_vid} function.\\\\
		\centerline{\includegraphics[scale=1]{IMG17}}\\\\
		\centerline{\textbf{Image 1: \code{detect\_vid} function and its arguments}}
		\item \code{0} means that OpenCV will open the webcam. The \code{desired\_classes}, as stated in Report No 1, is to define what we want to find by the program, ranging from 1 to 81, and 0 to find everything.
		\item If the code run successfully, the output will be somewhere in the line of this: \\\\
		 \centerline{\includegraphics[scale=0.3]{img18}}\\\\
		\centerline{\textbf{Image 2: Webcam window opened by OpenCV as well as the analysis}}
		\item Label and score will be added soon in the top left corner of the green bounding boxes.
	\end{itemize}
\subsection{GUI Analysis}
	\begin{itemize}
		\item Though we have decided to go the extra mile to do \textit{two} Wrapper for our core, we have run into quite a trouble
	\end{itemize}
	\subsubsection{Web}
		\begin{itemize}
		\item Here is what we have gotten so far:\\\\
		 \centerline{\includegraphics[scale=0.2]{img19}}\\\\
		\centerline{\textbf{Image 3: Our Web application's GUI}}
		\item As Minh is working on this, he has ran on some trouble relating the \code{File Not Found} error, specifically not being able to load the .js file for the webpage.
		\end{itemize}
	\subsubsection{Desktop Application}
		\begin{itemize}
			\item When Vuong tried to run the camera.py from the desktop application package that the mentors sent, it reports the following error: \code{ModuleNotFoundError: No module named 'PyQt5.QtMultimedia'}. StackOverFlow does not have much information about such error, and the only solution to fix it is to downgrade the \code{pyqt} library. Even after Vuong did that, nothing changed.
		\end{itemize}
\end{document}		